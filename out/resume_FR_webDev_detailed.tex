
%!TEX TS-program = xelatex
\documentclass[]{friggeri-cv}
\usepackage{afterpage}
\usepackage{hyperref}
\usepackage{color}
\usepackage{xcolor}
\usepackage{smartdiagram}
\usepackage{fontspec}
% if you want to add fontawesome package
% you need to compile the tex file with LuaLaTeX
% References:
%   http://texdoc.net/texmf-dist/doc/latex/fontawesome/fontawesome.pdf
%   https://www.ctan.org/tex-archive/fonts/fontawesome?lang=en
%\usepackage{fontawesome}
\usepackage{metalogo}
\usepackage{dtklogos}
\usepackage[utf8]{inputenc}
\usepackage{tikz}
\usepackage{multicol}
\usepackage{setspace}
\usepackage[document]{ragged2e}
%\usepackage{titlesec}
%\usepackage[skip=4pt, indent=0.0pt, parfill=4.0pt]{parskip}
\usetikzlibrary{mindmap,shadows}
\hypersetup{
    pdftitle={},
    pdfauthor={},
    pdfsubject={},
    pdfkeywords={},
    colorlinks=false,           % no lik border color
    allbordercolors=white       % white border color for all
}
\smartdiagramset{
    bubble center node font = \footnotesize,
    bubble node font = \footnotesize,
    % specifies the minimum size of the bubble center node
    bubble center node size = 0.5cm,
    %  specifies the minimum size of the bubbles
    bubble node size = 0.5cm,
    % specifies which is the distance among the bubble center node and the other bubbles
    distance center/other bubbles = 0.3cm,
    % sets the distance from the text to the border of the bubble center node
    distance text center bubble = 0.5cm,
    % set center bubble color
    bubble center node color = pblue,
    % define the list of colors usable in the diagram
    set color list = {lightgray, materialcyan, orange, green, materialorange, materialteal, materialamber, materialindigo, materialgreen, materiallime},
    % sets the opacity at which the bubbles are shown
    bubble fill opacity = 0.6,
    % sets the opacity at which the bubble text is shown
    bubble text opacity = 0.5,
}

\addbibresource{bibliography.bib}
\RequirePackage{xcolor}
\definecolor{pblue}{HTML}{0395DE}

%\titlespacing*{\section}
%{0pt}{12pt plus 4pt minus 2pt}{0pt plus 2pt minus 2pt}
%\titlespacing*{\subsection}
%{0pt}{12pt plus 4pt minus 2pt}{0pt plus 2pt minus 2pt}
%\titlespacing*{\subsubsection}
%{0pt}{12pt plus 4pt minus 2pt}{0pt plus 2pt minus 2pt}

\title{Yoann Chamillard -- Resume}
\author{Yoann Chamillard}
\date{25/5/2025}

\hypersetup{
  pdftitle={Yoann Chamillard -- Resume},
  pdfauthor={Yoann Chamillard},
  pdfsubject={Ingénieur Développeur Web \& Mobile Full Stack - Resume},
  pdfkeywords={profile webDev; resume; developer; software; engineer; C\# .Net; Python; Javascript; Node.js; Java; Kotlin; Android; Rest API; Git  GitLab; Docker; Jenkins; Selenium; Cron; Html-Css; MySQL; PostgreSQL; MongoDB; SQLite; Firebase; Bash; Jira; Regex},
  pdfcreator={LuaLaTeX},
  pdfproducer={LuaLaTeX}
}

\begin{document}

\header{Yoann}{Chamillard}
      {~~~~~~~~~~~~~~~~Ingénieur Développeur Web Full Stack}
      {}

\begin{aside}
\hspace{10mm}\includegraphics[scale=0.148]{res/img/Photo_CV.jpg}\section{Infos}
%33 ans
Permis A,B\vspace{2.5mm}
Paris, France\vspace{1.5mm}
Mobile à l'international
Autorisé à travailler au Canada (visa PVT)\vspace{2.5mm}
+33 670525552
\href{mailto:yoann.chamillard@gmail.com}{\small yoann.chamillard@gmail.com}\vspace{2.5mm}
\href{http://fr.linkedin.com/in/yoannchamillard}{LinkedIn\hspace{1.5mm}\includegraphics[scale=0.075]{res/img/hlink.png}}
\href{https://github.com/Nokheenig?tab=stars}{GitHub\hspace{1.5mm}\includegraphics[scale=0.075]{res/img/hlink.png}}\vspace{2.5mm}
\makebox[4.3cm][l]{\textbf{Français} }
\makebox[4.3cm][l]{\textbf{Anglais} (C1,Bulats)}
\makebox[4.3cm][l]{\textbf{Allemand} (B1)}
\makebox[4.3cm][l]{\textbf{Coréen} }\vspace{2.5mm}
\includegraphics[scale=0.40]{res/img/profileMap_FR.png}\vspace{2.5mm}
\section{CAO / FAO}
\includegraphics[scale=0.40]{res/img/5stars.png}\hspace{1.5mm}\textbf{TopSolid}
\includegraphics[scale=0.40]{res/img/5stars.png}\hspace{1.5mm}\textbf{Catia V5\&V6}
\includegraphics[scale=0.40]{res/img/4stars.png}\hspace{1.5mm}\textbf{Creo 4.0}
\includegraphics[scale=0.40]{res/img/3stars.png}\hspace{1.5mm}\textbf{Impression 3D}\section{PLM / PDM}
\includegraphics[scale=0.40]{res/img/4stars.png}\hspace{1.5mm}\textbf{TopSolid}
\includegraphics[scale=0.40]{res/img/4stars.png}\hspace{1.5mm}\textbf{NewPDM}
\includegraphics[scale=0.40]{res/img/3stars.png}\hspace{1.5mm}\textbf{Windchill}\section{Calcul / FEM}
\includegraphics[scale=0.40]{res/img/3stars.png}\hspace{1.5mm}\textbf{Ansys Wbench}
\includegraphics[scale=0.40]{res/img/3stars.png}\hspace{1.5mm}\textbf{Abaqus}
\includegraphics[scale=0.40]{res/img/2-5stars.png}\hspace{1.5mm}\textbf{Hyperworks}
\includegraphics[scale=0.40]{res/img/2-5stars.png}\hspace{1.5mm}\textbf{Hypermesh}
\vspace{2.5mm}%\begin{flushleft}
	\emph{Date: 25/5/2025} \hspace*{8mm}
    %{\tiny webDev} % profileId
	%\end{flushleft}
\end{aside}

\vspace*{-2.0mm}
\noindent\parbox{\linewidth}{
  \centering
  Ingénieur développeur logiciel avec une première expérience réussie, toujours curieux et à la recherche de nouveaux apprentissages et de nouveaux défis !
}
\vspace*{0.8mm}

\section{Compétences informatiques}
        \vspace*{-0.45cm}
        \setlength{\columnsep}{-0.3cm}
        \begin{flushleft}
        \begin{multicols}{3}
		\begin{itemize}
		
		\setlength{\itemsep}{5pt}
  		\setlength{\parskip}{0pt}
  		\setlength{\parsep}{0pt}
          
        
\item \large Application / Serveur \
\normalsize
\begin{flushleft}

\includegraphics[scale=0.40]{res/img/5stars.png}\hspace{1.5mm}\textbf{C\#}
\includegraphics[scale=0.40]{res/img/5stars.png}\hspace{1.5mm}\textbf{Python}\\Flask, Django, FastAPI, SQLAlchemy, PyMongo, PyTest, Numpy, Uvicorn, Pydantic, Requests\\\vspace{2mm}
\includegraphics[scale=0.40]{res/img/4stars.png}\hspace{1.5mm}\textbf{Node.js}\\Express.js, Bcrypt\\
\includegraphics[scale=0.40]{res/img/3stars.png}\hspace{1.5mm}\textbf{Java}
\includegraphics[scale=0.40]{res/img/3stars.png}\hspace{1.5mm}\textbf{\small Nginx,Apache}
\includegraphics[scale=0.40]{res/img/4stars.png}\hspace{1.5mm}\textbf{Rest API}\\OpenAPI standard\\
\end{flushleft}            

\columnbreak
\item \large Devops, CI/CD \
\normalsize
\begin{flushleft}

\includegraphics[scale=0.40]{res/img/5stars.png}\hspace{1.5mm}\textbf{Git / GitLab}
\includegraphics[scale=0.40]{res/img/4stars.png}\hspace{1.5mm}\textbf{Docker}
\includegraphics[scale=0.40]{res/img/3stars.png}\hspace{1.5mm}\textbf{Jenkins}
\includegraphics[scale=0.40]{res/img/4stars.png}\hspace{1.5mm}\textbf{Selenium}
\includegraphics[scale=0.40]{res/img/4stars.png}\hspace{1.5mm}\textbf{Cron}
\end{flushleft}            

\item \large Developpement web \
\normalsize
\begin{flushleft}

\includegraphics[scale=0.40]{res/img/4stars.png}\hspace{1.5mm}\textbf{HTML,CSS}\\SCSS/SASS\\
\includegraphics[scale=0.40]{res/img/4stars.png}\hspace{1.5mm}\textbf{JavaScript}\\Vue.js, Node.js\\
\end{flushleft}            

\item \large Authentification \
\normalsize
\begin{flushleft}

\includegraphics[scale=0.40]{res/img/4stars.png}\hspace{1.5mm}\textbf{JWT Auth}
\includegraphics[scale=0.40]{res/img/4stars.png}\hspace{1.5mm}\textbf{Firebase}
\end{flushleft}            

\columnbreak
\item \large Bases de données \
\normalsize
\begin{flushleft}

\includegraphics[scale=0.40]{res/img/4stars.png}\hspace{1.5mm}\textbf{MySQL}
\includegraphics[scale=0.40]{res/img/5stars.png}\hspace{1.5mm}\textbf{PostgreSQL}
\includegraphics[scale=0.40]{res/img/5stars.png}\hspace{1.5mm}\textbf{MongoDB}
\includegraphics[scale=0.40]{res/img/4stars.png}\hspace{1.5mm}\textbf{SQLite}
\includegraphics[scale=0.40]{res/img/3stars.png}\hspace{1.5mm}\textbf{Firebase}\\Storage/Firestore/Realtime\\
\end{flushleft}            

\item \large Autres \
\normalsize
\begin{flushleft}

\includegraphics[scale=0.40]{res/img/4stars.png}\hspace{1.5mm}\textbf{Bash}
\includegraphics[scale=0.40]{res/img/4stars.png}\hspace{1.5mm}\textbf{Jira}
\includegraphics[scale=0.40]{res/img/5stars.png}\hspace{1.5mm}\textbf{Regex}
\end{flushleft}            


        \end{itemize}
        \end{multicols}
        %\end{itemize}
        \end{flushleft} \normalsize
        \vspace*{-0.65cm}
\section{Expériences}
\vspace*{-0.25cm}

\begin{entrylist}
  \entry
    {09/24 - Auj.}
    {Ingénieur Développement}
    {TopSolid, \textit{Paris, FR}}
    {Développement de post-processeurs en:  C\# (.Net)}
\end{entrylist}
\vspace{-15pt}

\vspace{0.5mm}
\begin{itemize}
\setlength{\itemsep}{1pt}
\setlength{\parskip}{0pt}
\setlength{\parsep}{0pt}

\item Développement from-scratch de post-processeurs pour machines outils à commande numérique (CNC), robots d'usinage 6 axes et machines de découpe laser 2D/3D en C\# (.Net), G-code (Fanuc), ... + couche de customisation dans un langage propriétaire pour nos intégrateurs.
\item Rédaction de documentations et spécifications.
\item Essais et mise au point machine en condition réelle chez le client.
\item Rédaction de scripts et automatisation
\item Installation et configuration de machines virtuelles de test / simulation d'usinage
\item Support technique client et amélioration de post-processeurs existants.
\end{itemize}

\begin{entrylist}
  \entry
    {04/24 - 09/24}
    {Développeur web \& mobile}
    {Formation et recherche d'emploi, \textit{Paris, FR}}
    {Formation et réalisation de projets persos:\hspace*{8mm}Kotlin, Python, Javascript}
\end{entrylist}
\vspace{-15pt}

\vspace{0.5mm}
\begin{itemize}
\setlength{\itemsep}{1pt}
\setlength{\parskip}{0pt}
\setlength{\parsep}{0pt}

\item Formation masterclass Android 14 et développement Kotlin.
\item Développements Kotlin: appli de quiz, de dessin, appli de prise de note, appli météo, appli de chat avec notifications, clone de l'appli Uber
\item Développements Python: générateur de CV flexible en LateX multilingues se basant sur le standard JSON Resume
\item Développements Javascript: Appli web CRM pour m'assister dans la gestion de mes candidatures + scraping des offres d'emploi de multiples plateformes pour les centraliser et en réaliser la gestion dans une interface unique.
\item Montage d'un serveur personnel pour sauvegarder mes données avec redondance (raid) + déploiement et exposition d'applis et services
\end{itemize}

\begin{entrylist}
  \entry
    {10/22 - 03/24}
    {Développeur web \& mobile}
    {Synapsun, \textit{Lyon, FR}}
    {Développements fullstack web \& mobile Android:\hspace*{8mm}Flutter, Python}
\end{entrylist}
\vspace{-15pt}

\vspace{0.5mm}
\begin{itemize}
\setlength{\itemsep}{1pt}
\setlength{\parskip}{0pt}
\setlength{\parsep}{0pt}

\item Recueil du besoin, construction d'un cahier des charges et de spécifications fonctionnelles et techniques.
\item Développement d'un web scraper en Python
\item Conception de modèles de données
\item Rédaction de spécifications d'API au standard OpenAPI / Swagger
\item Mise en place d'une base de données produits MongoDB et d'une API Python FastAPI
\item Développement d'un client mobile multi-plateforme Flutter.
\item Gestion des accès par token JWT (Json Web Token)
\item Mise en place d'une base de données PostgreSQL et d'un back-end Python Django pour la gestion des utilisateurs et des devis.
\item Contact fournisseur/prestataire pour les travaux outsourcés
\end{itemize}

\begin{entrylist}
  \entry
    {11/19 - 04/22}
    {Ingénieur développement produit}
    {Böllhoff, \textit{Chambéry, FR}}
    {Conception produit mécatronique automobile}
\end{entrylist}
\vspace{-15pt}

\vspace{0.5mm}
\begin{itemize}
\setlength{\itemsep}{1pt}
\setlength{\parskip}{0pt}
\setlength{\parsep}{0pt}

\item Définition produit et produits liés au(x) process/essais (Géométrie, Simu/Calcul, Dimensionnement, Optimisation, Interfaces): Capteurs, Assemblages vissés sous précontrainte, Élément fixé sur caisse (réservoir, coffre de toit, ...), Élément de suspension/train (McPherson, Torsible, à lames), Machine spéciale (collage, préhension/commande pneumatique)
\item Mise en plan, Cotation fonctionnelle, Chaines de cotes (Arithmétiques, Quadratiques)
\item Assurer les jeux avec l'environnement (Archi, Thermiques,...)
\item Assurer la montabilité, la mise et maintien en position des produits par les moyens de montage
\item Rédaction de méthodes d’essai et réalisation d’essais de validation : Essais Labo, Sous-Système et Système, Essais de flexion, Essais de traction-compression
\item Plans d’Expérience/Taguchi/Design-Of-Experiments : Définition, Analyse
\item Extensométrie, Acquisition: Vérifier la tenue mécanique et la concordance des prototypes avec leur modèle simu à l’aide de jauges de déformation pelliculaires (HBM)
\item Plasturgie : Conception de pièces de boitier module avec fonctions d’étanchéité
\item Prototypage 3D FDM, Consultation fournisseur
\item Participation aux design review : présentation synthétique des problèmes, établissement de plans d’action, suivi de leur application
\end{itemize}

\begin{entrylist}
  \entry
    {11/16 - 09/19}
    {Ingénieur d'étude automobile}
    {Renault, \textit{Vélizy, FR}}
    {Conception \& architecture sous-caisse et compartiment moteur}
\end{entrylist}
\vspace{-15pt}

\vspace{0.5mm}
\begin{itemize}
\setlength{\itemsep}{1pt}
\setlength{\parskip}{0pt}
\setlength{\parsep}{0pt}

\item Définition produit (Géométrie, Interfaces): Tout type de pièce dans les zones d'architecture concernées.
\item Assurer les jeux avec l'environnement (Archi, Thermiques,...) avec les pièces fixes et mobiles
\item Assurer la montabilité, la mise et maintien en position des produits par les moyens de montage autos et manuels
\item Mise en plan, Cotation fonctionnelle, Chaines de cotes (Arithmétiques, Quadratiques)
\item Suivi de convergence du périmètre archi
\item Animation de groupe de travail (Pilotage de concepteurs français et à l'étranger)
\item Organisation et animation de design review : présentation synthétique des problèmes, établissement de plans d’action, suivi de leur application
\end{itemize}

\vspace*{-0.5cm}
\vspace*{0.45cm}
\section{Formations - Certifications}
\vspace*{-0.25cm}
\vspace{0.5mm}
    \begin{entrylist}
    \entry
        {09/22 - 08/23}
        {Bachelor Concepteur Développeur d'Applications}
        {EPSI, \textit{Lyon, FR}}
        {Spécialité Data/IA; \hspace{7mm} 09/23: Certification d'état CDA}
    \end{entrylist}
    \vspace{0.5mm}
\begin{entrylist}
\entry
    {06/22 - 08/22}
    {Formation d'Intégrateur-développeur web}
    {Epitech, \textit{Paris, FR}}
    {}
\end{entrylist}
\vspace{0.5mm}
    \begin{entrylist}
    \entry
        {09/13 - 08/16}
        {Diplôme d'Ingénieur en Systèmes Mécaniques}
        {UTT, \textit{Troyes, FR}}
        {Mineure: Conception et Industrialisation, en lien avec l’Environnement}
    \end{entrylist}
    \vspace{0.5mm}
\begin{entrylist}
\entry
    {09/11 - 06/13}
    {DUT Génie Mécanique et Productique}
    {IUT, \textit{Troyes, FR}}
    {}
\end{entrylist}
\vspace*{-0.7cm}
\begin{itemize}
\setlength{\itemsep}{1pt}
\setlength{\parskip}{0pt}
\setlength{\parsep}{0pt}
\item Coupe de France de robotique: Conception, fabrication et programmation d’un robot
\end{itemize}

\end{document}